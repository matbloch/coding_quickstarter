\documentclass[12pt]{article}
\usepackage{amsmath}
\usepackage{amssymb}
\usepackage{times}
\newcommand{\A}{\textit{\textbf{A}}}
\newcommand{\aff}{\textit{\textbf{aff}}}
\newcommand{\adj}{\textrm{adj}}
\newcommand{\Aff}{\textit{\textbf{Aff}}}
\newcommand{\B}{\textit{\textbf{B}}}
\newcommand{\C}{\textit{\textbf{C}}}
\newcommand{\co}{\textit{\textbf{co}}}
\newcommand{\CO}{\textit{\textbf{CO}}}
\newcommand{\CP}{\textit{\textbf{CP}}}
\newcommand{\CR}{\textit{\textbf{CR}}}
\newcommand{\Dz}{\textit{\textbf{D}}}
\newcommand{\ddo}{\textit{\textbf{do}}}
\newcommand{\DO}{\textit{\textbf{DO}}}
\newcommand{\DP}{\textit{\textbf{DP}}}
\newcommand{\E}{\textit{\textbf{E}}}
\newcommand{\End}{\textit{\textbf{End}}}
\newcommand{\F}{\textit{\textbf{F}}}
\newcommand{\G}{\textit{\textbf{G}}}
\newcommand{\gl}{\textit{\textbf{gl}}}
\newcommand{\Gl}{\textit{\textbf{Gl}}}
\newcommand{\Hh}{\textit{\textbf{H}}}
\newcommand{\I}{\textit{\textbf{I}}}
\newcommand{\J}{\textit{\textbf{J}}}
\newcommand{\K}{\textit{\textbf{K}}}
\newcommand{\Ll}{\textit{\textbf{L}}}
\newcommand{\lz}{\textit{\textbf{l}}}
\newcommand{\M}{\textit{\textbf{M}}}
\newcommand{\N}{\textit{\textbf{N}}}
\newcommand{\oo}{\textit{\textbf{o}}}
\newcommand{\Oo}{\textit{\textbf{O}}}
\newcommand{\p}{\textit{\textbf{p}}}
\newcommand{\Pp}{\textit{\textbf{P}}}
\newcommand{\pl}{\textit{\textbf{pl}}}
\newcommand{\Pl}{\textit{\textbf{Pl}}}
\newcommand{\q}{\textit{\textbf{q}}}
\newcommand{\Q}{\textit{\textbf{Q}}}
\newcommand{\R}{\textit{\textbf{R}}}
\newcommand{\Ss}{\textit{\textbf{S}}}
\newcommand{\Sl}{\textit{\textbf{Sl}}}
\newcommand{\SO}{\textit{\textbf{SO}}}
\newcommand{\T}{\textit{\textbf{T}}}
\newcommand{\U}{\textit{\textbf{U}}}
\newcommand{\V}{\textit{\textbf{V}}}
\newcommand{\W}{\textit{\textbf{W}}}
\newcommand{\w}{\textit{\textbf{w}}}
\newcommand{\X}{\textit{\textbf{X}}}
\newcommand{\Y}{\textit{\textbf{Y}}}
\newcommand{\Z}{\textit{\textbf{Z}}}





     
    














\thispagestyle{empty}
\begin{document}
\noindent \section*{Internet Calculus II
\hspace{5pt} Homework 3 Solutions\\Sunday November 4th 2001}
\subsection*{Question 1}
Find the inverse of the following matrices, $A$, $B$, $C$ and $D$:
\[ A\hspace{4pt} = \hspace{4pt} \begin{array}{|cc|}2&-5\\-3&8\end{array},\]
\[ B\hspace{4pt} = \hspace{4pt}\begin{array}{|cc|}3&1\\7&2\end{array},\]
\[ C = AB, \]
\[ D = BA.\]
Also compute the matrices $E = A^{-1}B^{-1}$ and $F = B^{-1}A^{-1}$ and discuss your results.\\
\\
We use the formula for the inverse of a matrix $A$: 
$A^{-1} = \frac{1}{\det(A)}\adj(A)$.
\begin{itemize}\item $A^{-1}$:
\[ A\hspace{4pt} = \hspace{4pt} \begin{array}{|cc|}2&-5\\-3&8\end{array},\]
\[ \adj(A) = \hspace{4pt} \begin{array}{|cc|}8&5\\3&2\end{array}, \]
\[ \det(A) = 2(8) - (-5)(-3) = 16 - 15 = 1,\]
\[ A^{-1} = \frac{1}{\det(A)}\adj(A) =  \hspace{4pt} \begin{array}{|cc|}8&5\\3&2\end{array}.\]
\item $B^{-1}$:
\[ B\hspace{4pt} = \hspace{4pt}\begin{array}{|cc|}3&1\\7&2\end{array},\]
\[ \adj(B) = \hspace{4pt} \begin{array}{|cc|}2&-1\\-7&3\end{array}, \]
\[ \det(B) = 3(2) - 1(7) = 6 - 7 = - 1,\]
\[ B^{-1} = \frac{1}{\det(B)}\adj(B) =  \hspace{4pt} \begin{array}{|cc|}-2&1\\7&-3\end{array}.\]
\item $C^{-1}= (AB)^{-1}$.
\[ C = AB =  \hspace{4pt} \begin{array}{|cc|}2&-5\\-3&8\end{array} \hspace{4pt} \begin{array}{|cc|}3&1\\7&2\end{array}\]
\[ =  \hspace{4pt} \begin{array}{|cc|}6 - 35&2 - 10\\-9 + 56&- 3 + 16\end{array}\]
\[ =  \hspace{4pt} \begin{array}{|cc|}-29&-8\\47&13\end{array}\]
\[ \det(C) = (-29)(13) - (-8)(47) = - 377 + 376 = -1, \]
\[ C^{-1} = \frac{1}{\det(C)} \adj(C) = - \adj(C) = \hspace{4pt} \begin{array}{|cc|}-13&-8\\47&29\end{array}\]
\item $D^{-1}= (BA)^{-1}$.
\[ D = BA =  \hspace{4pt} \begin{array}{|cc|}3&1\\7&2\end{array} \hspace{4pt} \begin{array}{|cc|}2&-5\\-3&8\end{array}\]
\[ =  \hspace{4pt} \begin{array}{|cc|}6 - 3&- 15 + 8\\14 - 6&- 35 + 16\end{array}\]
\[ =  \hspace{4pt} \begin{array}{|cc|}3&-7\\8&-19\end{array}\]
\[ \det(D) = 3(-19) - (-7)(8) = - 57 + 56 = -1, \]
\[ D^{-1} = \frac{1}{\det(D)} \adj(D) = - \adj(D) = \hspace{4pt} \begin{array}{|cc|}19&-7\\8&-3\end{array}\]
\end{itemize}
Next we have:
\[ E = A^{-1}B^{-1} =  \hspace{4pt} \begin{array}{|cc|}8&5\\3&2\end{array} \hspace{4pt} \begin{array}{|cc|}-2&1\\7&-3\end{array}\]
\[ =  \hspace{4pt} \begin{array}{|cc|}-16 + 35&8- 15\\- 6 + 14&3 - 6\end{array}\]
\[ =  \hspace{4pt} \begin{array}{|cc|}19&-7\\8&-3\end{array}\]
\[ F = B^{-1}A^{-1} = \hspace{4pt}\begin{array}{|cc|}-2&1\\7&-3\end{array}  \hspace{4pt} \begin{array}{|cc|}8&5\\3&2\end{array}\]
\[ =  \hspace{4pt} \begin{array}{|cc|}-16 + 3&-10 + 2\\56 - 9&35 - 6\end{array}\]
\[ =  \hspace{4pt} \begin{array}{|cc|}-13&-8\\47&29\end{array}\]
We see that $E = A^{-1}B^{-1} = D = (BA)^{-1}$ and $F = B^{-1}A^{-1} = C = (AB)^{-1}$.\\
It appears that taking the inverse of a product gives the produc tof the inverses in the opposite order.\\
We may see this as follows:\\
The inverse of a matrix $M$, if it exists is the unique matrix $N$ such that $MN = NM = I$.\\
Put $M = AB$ and $N = (AB)^{-1}$.\\
Then by definition we have $MN = NM = I$.\\
Put $N' = B^{-1}A^{-1}$.\\
Then we have:
\[ MN' = AB(B^{-1}A^{-1}) = A(BB^{-1})A^{-1} = AIA^{-1} = AA^{-1} = I, \]
\[ N'M = B^{-1}A^{-1}AB = B^{-1}IB = B^{-1}B = I.\]
So $N'$ is also an inverse for $M$.\\
So $N = N'$ and $(AB)^{-1} = B^{-1}A^{-1}$, as required.\\
More generally, by a similar argument, we have:
\[ (A_1A_2A_3\dots A_n)^{-1} = A_n^{-1}\dots A^{-1}_3A^{-1}_2A^{-1}_1.\]  



\subsection*{Question 2}
Would you expect that the inverse of the square of a matrix is the square of the inverse:
\[ (G^{-1})^2 = (G^2)^{-1}?\]
Answer: yes.\\
Using the result of the last question, we have:
\[ (G^2)^{-1} = (GG)^{-1} = G^{-1}G^{-1} = (G^{-1})^2.\]

Illustrate your answer, with the case of the matrix:
\[ G = \begin{array}{|cc|}3&-4\\-2&3\end{array}.\]
\[ G^2 = GG =  \hspace{4pt} \begin{array}{|cc|}3&-4\\-2&3\end{array} \hspace{4pt} \begin{array}{|cc|}3&-4\\-2&3\end{array}\]
\[ =  \hspace{4pt} \begin{array}{|cc|}9 + 8&-12 - 12\\-6 - 6&8 + 9\end{array}\]
\[ =  \hspace{4pt} \begin{array}{|cc|}17&-24\\-12&17\end{array}\]
\[ \det(G^2) = 17(17)-(-24)(-12) = 289 - 288 = 1, \]
\[ (G^2)^{-1} = \frac{1}{\det(G^2)}\adj(G^2) = \adj(G^2) = \hspace{4pt} \begin{array}{|cc|}17&24\\12&17\end{array}.\]
Next we have:
\[ \det(G) = 3(3) - (-2)(-4) = 9 - 8 = 1, \]
\[ (G)^{-1} = \frac{1}{\det(G)}\adj(G) = \adj(G) = \hspace{4pt} \begin{array}{|cc|}3&4\\2&3\end{array}.\]
\[ (G^{-1})^2 = G^{-1}G^{-1} =  \hspace{4pt} \begin{array}{|cc|}3&4\\2&3\end{array} \hspace{4pt} \begin{array}{|cc|}3&4\\2&3\end{array}\]
\[ =  \hspace{4pt} \begin{array}{|cc|}9 + 8&12 + 12\\6 + 6&8 + 9\end{array}\]
\[ =  \hspace{4pt} \begin{array}{|cc|}17&24\\12&17\end{array}\]
So we see that indeed $(G^2)^{-1} = G^{-1})^2$, as expected.
\subsection*{Question 3}
Consider the matrix equation:
\[ AX = B, \]
\[ A\hspace{4pt} = \hspace{4pt} \begin{array}{|cc|}2&-5\\-3&8\end{array},\]
\[ X\hspace{4pt} = \hspace{4pt}\begin{array}{|cc|}x&p\\y&q\end{array},\]
\[ B\hspace{4pt} = \hspace{4pt}\begin{array}{|cc|}3&1\\7&2\end{array}.\]
By writing out these equations, show that they decouple into a pair of linear equations for the unknowns $x$ and $y$ and another pair for the unknowns $p$ and $q$, with the same matrix of coefficients, the matrix $A$.\\
By using the inverse of the matrix $A$, solve the system $AX = B$ (i.e. multiply both sides on the left by $A^{-1}$) and show that your solution does give solutions to the decoupled equations.\\
Can you also see how to solve the system $XA = B$?
We need $B = AX$:
\[  \hspace{4pt}\begin{array}{|cc|}3&1\\7&2\end{array}\hspace{4pt} = \hspace{4pt} \begin{array}{|cc|}2&-5\\-3&8\end{array} \hspace{4pt}\begin{array}{|cc|}x&p\\y&q\end{array},\]
\[ = \begin{array}{|cc|}2x -5y&2p - 5q\\-3x + 8y&-3p + 8q\end{array}.\]
Comparing entries we find the two decoupled systems, as required:
\[ 3 = 2x - 5y, \hspace{10pt} 7 = -3x + 8y, \]
\[ 1 = 2p - 5q, \hspace{10pt} 2 = -3p + 8q.\]
Multiplying the equation $AX = B$ on the left by $A^{-1}$ gives the equation:
\[ A^{-1}AX = A^{-1}B, \]
\[ IX = A^{-1}B.\]
\[ X = A^{-1}B.\]
Conversely, if $X = A^{-1}B$, then we have:
\[ AX = A(A^{-1}B) = (AA^{-1})B = IB = B.\]
This shows that provided $A^{-1}$ exists the complete solution of the matrix equation $AX = B$ is $X = A^{-1}B$.\\
So here we get:
\[ X = A^{-1}B = \frac{1}{\det(A)}\adj(A)B\]
\[ = \frac{1}{2(8) - (-5)(-3)}\hspace{4pt}\begin{array}{|cc|}8&5\\3&2\end{array}  \hspace{4pt}\begin{array}{|cc|}3&1\\7&2\end{array}\]
\[ =  \hspace{4pt}\begin{array}{|cc|}24 + 35&8+10\\9 + 14&3 + 4\end{array}\]
\[ = \hspace{4pt}\begin{array}{|cc|}59&18\\23&7\end{array}\]
The first column of $X$ gives the solution $x = 59$, $y = 23$ of the first system.\\
The second column of $X$ gives the solution $p = 18$, $y = 7$ of the second system.\\
One can check directly that these are indeed the solutions.\\
By similar arguments to the above, the solution to the equation $XA = B$ is $X = BA^{-1}$.\\
So here we get:
\[ X = \hspace{4pt}\begin{array}{|cc|}3&1\\7&2\end{array}  \hspace{4pt}\begin{array}{|cc|}8&5\\3&2\end{array}\]
\[ =  \hspace{4pt}\begin{array}{|cc|}24 + 3&15+2\\56 + 6&35 + 4\end{array}\]
\[ = \hspace{4pt}\begin{array}{|cc|}27&17\\62&39\end{array}\] 
One can check directly that $X$ obeys the equation $XA = B$, as required.
\subsection*{Question 4}
Using matrix techniques, find the point of intersection of the following lines $L$ and $M$:
\[ L: 4x - 5y = 7, \hspace{10pt} M: 3x - 2y = 14.\]
At the point of intersection, both equations hold.\\
We may write them in matrix form:
\[  \begin{array}{|cc|}4&-5\\3&-2\end{array}\hspace{4pt}\begin{array}{|c|}x\\y\end{array} \hspace{4pt} = \hspace{4pt}\begin{array}{|c|}7\\14\end{array},\]
\[ AX = B,\]
\[ A =  \begin{array}{|cc|}4&-5\\3&-2\end{array}, \hspace{4pt}X = \hspace{4pt}\begin{array}{|c|}x\\y\end{array}, \hspace{4pt}B = \hspace{4pt}\begin{array}{|c|}7\\14\end{array}.\]
The solution is then:
\[ X = A^{-1}B = \frac{1}{\det(A)}\adj(A) B \]
\[ = \frac{1}{4(-2) - (-5)(3)}\hspace{4pt} \begin{array}{|cc|}-2&5\\-3&4\end{array} \hspace{4pt}\begin{array}{|c|}7\\14\end{array}\]
\[ = \frac{1}{-8 + 15} \hspace{4pt}\begin{array}{|c|}-14 + 70\\-21 + 56\end{array}\]
\[ = \frac{1}{7} \hspace{4pt}\begin{array}{|c|}56\\35\end{array}\]
\[ = \hspace{4pt}\begin{array}{|c|}8\\5\end{array}\]
So the meeting point is $(8, 5)$.\\
It is easily checked that this point lies on both lines, as required.
\subsection*{Question 5}
Let $P$ and $Q$ be the following lines given in parametric form:
\[P: [x,y] = [2, -3] + t[4, 1], \]
\[ Q: [x,y] = [5, -6] + s[3, 2].\]
Show that the problem of finding the point of intersection gives two linear equations for $s$ and $t$.\\
Write these in matrix form and solve them, using matrix techniques.\\\\
At the point of intersection, we have the same $x$ and $y$ for both lines, giving the equations:
\[ 2 + 4t = 5 + 3s, \hspace{10pt} -3  + t = -6 + 2s, \]
\[ 3s - 4t = - 3, \hspace{10pt} 2s - t = 3, \]
\[  \begin{array}{|cc|}3&-4\\2&-1\end{array}\hspace{4pt}\begin{array}{|c|}s\\t\end{array} \hspace{4pt} = \hspace{4pt}\begin{array}{|c|}-3\\3\end{array},\]
\[ AX = B,\]
\[ A =  \begin{array}{|cc|}3&-4\\2&-1\end{array}, \hspace{4pt}X = \hspace{4pt}\begin{array}{|c|}s\\t\end{array}, \hspace{4pt}B = \hspace{4pt}\begin{array}{|c|}-3\\3\end{array}.\]
The solution is then:
\[ X = A^{-1}B = \frac{1}{\det(A)}\adj(A) B \]
\[ = \frac{1}{3(-1) - (-4)(2)}\hspace{4pt} \begin{array}{|cc|}-1&4\\-2&3\end{array} \hspace{4pt}\begin{array}{|c|}-3\\3\end{array}\]
\[ = \frac{1}{-3 + 8} \hspace{4pt}\begin{array}{|c|}3 + 12\\6 + 9\end{array}\]
\[ = \frac{1}{5} \hspace{4pt}\begin{array}{|c|}15\\15\end{array}\]
\[ = \hspace{4pt}\begin{array}{|c|}3\\3\end{array}\]
So the meeting point is given by $s = 3$ and $t =3$.\\
Indeed putitng $s = 3$ and $t = 3$ in the equations of $P$ and $Q$ gives the same meeting point:
\[ [2, -3] + 3[4, 1] = [14, 0] = [5, -6] + 3[3, 2].\]
So the lines $P$ and $Q$ meet at the point $[14, 0]$.
\end{document}
\end






